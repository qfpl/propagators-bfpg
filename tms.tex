\textslide{
  What happens when we hit contradiction?
  
  \pnl
  
  {\Huge{:(}}

}

\textslide{
  \Large{
    If we track the provenance of information, \\
    we can help identify the source of contradiction

    \pnl

    Then we can keep track of which subsets of the information are consistent

    \nl

    and which are inconsistent
  }
}

\textslideleft{
  \begin{columns}[T]
    \column{0.05\textwidth}
    \column{0.35\textwidth}
    \Large

    $ [2,5] \cap [3,7] \cap [6,9] = []$

    \pnl

    $[2,5] \cap [3,7] = [3,5]$

    \pnl

    $[3,7] \cap [6,9] = [6,7]$

    \pnl

    $[2,5] \cap [6,9] = []$

  \column{0.3\textwidth}
  \pause
  \centering
    Consistent subsets:

    $\{\}$
    
    $\{[2,5]\}$
    
    $\{[3,7]\}$
    
    $\{[6,9]\}$

    $\{[2,5], [3,7]\}$

    $\{[3,7], [6,9]\}$

    \pnl

    Maximal consistent subsets:

    $\{[2,5], [3,7]\}$

    $\{[3,7], [6,9]\}$

    \column{0.3\textwidth}
    \pause
    \centering
    Inconsistent subsets:

    $\{[2,5], [6,9]\}$
    $\{[2,5], [3,7], [6,9]\}$

    \pnl

    Minimal inconsistent subsets:

    $\{[2,5], [6,9]\}$
  \end{columns}
}

\textslide{
  \Large{
    This concept is something called a {\it Truth Management System}
  }
}

\textslide{
  Now that we can handle contradiction, we can make guesses!

  \pnl

  This lets us encode search problems easily

  \nl

  \def\svgwidth{0.6\columnwidth}
  \input{coloured-map.pdf_tex}
}
