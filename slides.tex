\documentclass[usenames,dvipsnames,svgnames,table,aspectratio=169,mathserif]{beamer}

\mode<presentation> {

%\usetheme{default}
\usetheme{Madrid}

\setbeamertemplate{footline} % To remove the footer line in all slides uncomment this line

\setbeamertemplate{navigation symbols}{} % To remove the navigation symbols from the bottom of all slides uncomment this line
}

\usepackage{graphicx} % Allows including images
\usepackage{booktabs} % Allows the use of \toprule, \midrule and \bottomrule in tables
\usepackage{hyperref}
\usepackage{apacite}
\usepackage{fancyvrb}
\usepackage{color}
\usepackage{alltt}
\usepackage{listings}
\usepackage{framed}
\usepackage{courier}
\usepackage{minted}
\usepackage{epstopdf}
\usepackage{xifthen}

\hypersetup{colorlinks=false}

\setbeamertemplate{bibliography entry title}{}
\setbeamertemplate{bibliography entry location}{}
\setbeamertemplate{bibliography entry note}{}
\setbeamertemplate{itemize items}[default]
\setbeamertemplate{enumerate items}[default]
\beamertemplatenavigationsymbolsempty
\setbeamertemplate{footline}{}

\newminted{haskell}{}
\newminted{java}{}
\newminted{scala}{}

\definecolor{g}{RGB}{0,100,0}
\newcommand{\highlight}[1]{\colorbox{yellow}{#1}}
\newcommand{\nega}[1]{\colorbox{yellow}{#1}}
\newcommand{\posi}[1]{\colorbox{green}{#1}}
\newcommand{\nl}{\vspace{\baselineskip}}
\newcommand{\pnl}{\pause \nl}

\newcommand{\imageslide}[2][1]{{
\begin{frame}\begin{center}
\includegraphics[scale=#1]{#2}
\end{center}\end{frame}
}}

%%----------------------------------------------------------------------------------------
%	TITLE PAGE
%----------------------------------------------------------------------------------------

\title[Propagators]{Propagators: An Introduction} % The short title appears at the bottom of every slide, the full title is only on the title page
\titlegraphic{\includegraphics[scale=0.2]{data61.eps}}
\author{George Wilson} % Your name
\institute[] % Your institution as it will appear on the bottom of every slide, may be shorthand to save space
{
Data61/CSIRO\\ % Your institution for the title page
\medskip
\href{george.wilson@data61.csiro.au}{george.wilson@data61.csiro.au} % Your email address
}
\date{\today} % Date, can be changed to a custom date

\begin{document}


%%%%%
%%%%% Intro section
%%%%%


\begin{frame}
\titlepage % Print the title page as the first slide
\end{frame}


\begin{frame}

\begin{columns}
  \begin{column}{0.5\textwidth}
    \begin{center}
      \includegraphics[scale=0.015]{what-are-birds.jpg}

      \nl

      What?
    \end{center}
  \end{column}
  \begin{column}{0.5\textwidth}
    \begin{center}
      \includegraphics[scale=0.3]{for-what-purpose.jpg}

      \nl

      Why?
    \end{center}
  \end{column}
\end{columns}

\end{frame}


\begin{frame}
The propagator model is a model of computation

\nl

We model computations as {\it propagator networks}
\end{frame}


\begin{frame}

A propagator network comprises
\begin{itemize}
\item cells
\item propagators
\item connections between cells and propagators
\end{itemize}

\end{frame}


\imageslide{diagrams/cell1.pdf}
\imageslide{diagrams/cell2.pdf}
\imageslide{diagrams/prop.pdf}

\imageslide{diagrams/add1.pdf}
\imageslide{diagrams/add2.pdf}
\imageslide{diagrams/add3.pdf}
\imageslide{diagrams/add4.pdf}


\begin{frame}
  \begin{center}
    \begin{LARGE}
      $z \leftarrow x + y$
    \end{LARGE}
  \end{center}
\end{frame}


\begin{frame}
  \begin{center}
    \begin{LARGE}
      $z = x + y$
    \end{LARGE}
  \end{center}
\end{frame}


\begin{frame}
  \begin{center}
    \begin{LARGE}
      $7 = x + 4$
    \end{LARGE}
  \end{center}
\end{frame}


\begin{frame}
  \begin{center}
    \begin{LARGE}
      $z \leftarrow x + y$

      \nl

      $x \leftarrow z - y$

      \nl

      $y \leftarrow z - x$

    \end{LARGE}
  \end{center}
\end{frame}


\imageslide{diagrams/badd1.pdf}
\imageslide{diagrams/badd2.pdf}
\imageslide{diagrams/badd3.pdf}
\imageslide{diagrams/badd4.pdf}


\begin{frame}
Propagators let us express multidirectional relationships
\end{frame}

\imageslide[0.8]{diagrams/celsius1.pdf}
\imageslide[0.8]{diagrams/celsius2.pdf}
\imageslide[0.8]{diagrams/celsius3.pdf}
\imageslide[0.8]{diagrams/celsius4.pdf}
\imageslide[0.65]{diagrams/celsius5.pdf}


\end{document} 

