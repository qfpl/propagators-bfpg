\documentclass[usenames,dvipsnames,svgnames,table,aspectratio=1610,mathserif]{beamer}

\mode<presentation> {

%\usetheme{default}
\usetheme{Madrid}

\setbeamertemplate{footline} % To remove the footer line in all slides uncomment this line

\setbeamertemplate{navigation symbols}{} % To remove the navigation symbols from the bottom of all slides uncomment this line
}

\usepackage{graphicx} % Allows including images
\usepackage{booktabs} % Allows the use of \toprule, \midrule and \bottomrule in tables
\usepackage{hyperref}
\usepackage{apacite}
\usepackage{fancyvrb}
\usepackage{color}
\usepackage{alltt}
\usepackage{listings}
\usepackage{framed}
\usepackage{courier}
\usepackage{minted}
\usepackage{epstopdf}
\usepackage{xifthen}
\usepackage[utf8]{inputenc}
\usepackage[T1]{fontenc}
\usepackage{textcomp}
\usepackage{gensymb}
\usepackage{svg}
\usepackage{pdfpages}

\hypersetup{colorlinks=false}

\setbeamertemplate{bibliography entry title}{}
\setbeamertemplate{bibliography entry location}{}
\setbeamertemplate{bibliography entry note}{}
\setbeamertemplate{itemize items}[circle]
\setbeamertemplate{enumerate items}[circle]
\beamertemplatenavigationsymbolsempty
\setbeamertemplate{footline}{}

%%%%%%%%%%%%%%%%%%%%%%%%%%%%%%%%%%%%%%%%%%%%%%%%%
% BACKGROUND COLOUR FOR BFPG
% CHANGE THIS BEFORE GIVING THIS TALK ANYWHERE ELSE
\definecolor{cream}{RGB}{255, 255, 204}
\setbeamercolor{background canvas}{bg=cream}
%%%%%%%%%%%%%%%%%%%%%%%%%%%%%%%%%%%%%%%%%%%%%%%%%


\newminted{haskell}{}

\definecolor{g}{RGB}{0,100,0}
\newcommand{\highlight}[1]{\colorbox{yellow}{#1}}
\newcommand{\nega}[1]{\colorbox{yellow}{#1}}
\newcommand{\posi}[1]{\colorbox{green}{#1}}
\newcommand{\nl}{\vspace{\baselineskip}}
\newcommand{\pnl}{\pause \nl}

\graphicspath{{diagrams/}}

\newcommand{\textslide}[1]{{
\begin{frame}
\begin{center}

#1

\end{center}
\end{frame}
}}

\newcommand{\textslideleft}[1]{{
\begin{frame}

#1

\end{frame}
}}

\newcommand{\latticeinfoslide}[1]{{
\begin{frame}
\begin{columns}
\column{0.7\textwidth}
\includegraphics[scale=0.65]{#1}
\column{0.3\textwidth}
\includegraphics[scale=0.5]{more-information.pdf}
\end{columns}
\end{frame}
}}

\newcommand{\codeslide}[1]{{
\begin{frame}[fragile]
\begin{haskellcode}
#1
\end{haskellcode}
\end{frame}
}}

\newcommand{\imageslide}[2][1]{{
\begin{frame}\begin{center}
\includegraphics[scale=#1]{#2}
\end{center}\end{frame}
}}

\newcommand{\imageslideleft}[2][1]{{
\begin{frame}
\includegraphics[scale=#1]{#2}
\end{frame}
}}

\newcommand{\imagetextslide}[3][1]{{
\begin{frame}\begin{center}

{#3}

\includegraphics[scale=#1]{#2}
\end{center}\end{frame}
}}

\newcommand{\svgslide}[1]{{
\begin{frame}
\begin{center}
\includesvg{diagrams/#1}
\end{center}
\end{frame}
}}

\newcommand{\ctof}{{
\LARGE $\degree F = \degree C \times \frac{9}{5} + 32$
}}

\newcommand{\ftoc}{{
\LARGE $\degree C = (\degree F - 32) \div \frac{9}{5}$
}}

%%----------------------------------------------------------------------------------------
%	TITLE PAGE
%----------------------------------------------------------------------------------------

\title[Propagators]{Propagators: An Introduction} % The short title appears at the bottom of every slide, the full title is only on the title page
\titlegraphic{\includegraphics[scale=0.2]{data61.eps}}
\author{George Wilson} % Your name
\institute[] % Your institution as it will appear on the bottom of every slide, may be shorthand to save space
{
Data61/CSIRO\\ % Your institution for the title page
\medskip
\href{george.wilson@data61.csiro.au}{george.wilson@data61.csiro.au} % Your email address
}
\date{\today} % Date, can be changed to a custom date

\begin{document}


%%%%%
%%%%% Intro section
%%%%%


\begin{frame}
\titlepage % Print the title page as the first slide
\end{frame}


\begin{frame}

\begin{columns}
  \begin{column}{0.5\textwidth}
    \begin{center}
      \includegraphics[scale=0.015]{what-are-birds.jpg}

      \nl

      What?
    \end{center}
  \end{column}
  \begin{column}{0.5\textwidth}
    \begin{center}
      \includegraphics[scale=0.3]{for-what-purpose.jpg}

      \nl

      Why?
    \end{center}
  \end{column}
\end{columns}

\end{frame}


%%%%%
%%%%% History
%%%%%


\begin{frame}

\begin{columns}
\begin{column}{0.5\textwidth}
Beginnings as early as the 1970's at MIT
\begin{itemize}
  \item Guy L. Steele Jr. 
  \item Gerald J. Sussman
  \item Richard Stallman
\end{itemize}

\nl

More recently:
\begin{itemize}
  \item Alexey Radul
\end{itemize}
\end{column}
\begin{column}{0.5\textwidth}

\begin{figure}
\centering
\def\svgwidth{\columnwidth}
\input{circuit.pdf_tex}
\end{figure}

\end{column}
\end{columns}

\end{frame}


\begin{frame}[fragile]

\begin{verbatim}
(define (map f xs)
  (cond ((null? xs) '())
        (else (cons (f (car xs))
                    (map f (cdr xs)))))))
\end{verbatim}
\end{frame}


\begin{frame}
\begin{columns}
\begin{column}{0.005\textwidth}
\end{column}
\begin{column}{0.4\textwidth}
And then
\begin{itemize}
\item Edward Kmett
\end{itemize}
\nl
\nl
\includegraphics[scale=0.2]{haskell.png}
\end{column}
\begin{column}{0.5\textwidth}
\includegraphics[scale=0.4]{powerset.pdf}

\nl
\nl

{\LARGE
  $x \le y \implies f(x) \le f(y)$
}
\end{column}
\end{columns}
\end{frame}


\begin{frame}
They're related to many areas of research, including:

\begin{itemize}
\item Logic programming (particularly Datalog)
\item Constraint solvers
\item Conflict-Free Replicated Datatypes
\item LVars
\item Programming language theory
\item And Spreadsheets!
\end{itemize}

They have advantages:

\begin{itemize}
\item are extremely expressive
\item lend themselves to parallel and distributed evaluation
\item allow different strategies of problem-solving to cooperate
\end{itemize}
\end{frame}


%%%%%
%%%%% What and why
%%%%%


\begin{frame}

\begin{center}
{\Huge Propagators}
\end{center}

\end{frame}


\begin{frame}
The {\it propagator model} is a model of computation

We model computations as {\it propagator networks}

\pnl

A propagator network comprises
\begin{itemize}
\item cells
\item propagators
\item connections between cells and propagators
\end{itemize}

\end{frame}


\imageslide{cell1.pdf}
\imageslide{cell2.pdf}
\imageslide{prop.pdf}

\imageslide{upper1.pdf}
\imageslide{upper2.pdf}
\imageslide{upper3.pdf}

%%%%% bidirectionality

\imageslide{add1.pdf}
\imageslide{add2.pdf}
\imageslide{add3.pdf}
\imageslide{add4.pdf}

\begin{frame}
  \begin{center}
    \begin{LARGE}
      $z \leftarrow x + y$
    \end{LARGE}
  \end{center}
\end{frame}


\begin{frame}
  \begin{center}
    \begin{LARGE}
      $z = x + y$
    \end{LARGE}
  \end{center}
\end{frame}


\begin{frame}
  \begin{center}
    \begin{LARGE}
      $7 = x + 4$
    \end{LARGE}
  \end{center}
\end{frame}


\begin{frame}
  \begin{center}
    \begin{LARGE}
      $7 = 3 + 4$
    \end{LARGE}
  \end{center}
\end{frame}


\begin{frame}
  \begin{center}
    \begin{LARGE}
      $z = x + y$
    \end{LARGE}
  \end{center}
\end{frame}


\begin{frame}
  \begin{center}
    \begin{LARGE}
      $z \leftarrow x + y$

      \nl

      $x \leftarrow z - y$

      \nl

      $y \leftarrow z - x$

    \end{LARGE}
  \end{center}
\end{frame}


\imageslide{badd1.pdf}
\imageslide{badd2.pdf}
\imageslide{badd3.pdf}
\imageslide{badd4.pdf}


\begin{frame}
\begin{center}
{\LARGE Propagators let us express bidirectional relationships!}
\end{center}
\end{frame}

\imagetextslide[0.8]{celsius1.pdf}{\ctof}
\imagetextslide[0.8]{celsius2.pdf}{\ctof}
\imagetextslide[0.8]{celsius3.pdf}{\ctof}
\imagetextslide[0.8]{celsius4.pdf}{\ctof}
\imagetextslide[0.65]{celsius5.pdf}{\ctof \\ \nl \ftoc}
\imagetextslide[0.65]{celsius6.pdf}{\ctof \\ \nl \ftoc}
\imagetextslide[0.65]{celsius7.pdf}{\ctof \\ \nl \ftoc}
\imagetextslide[0.65]{celsius8.pdf}{\ctof \\ \nl \ftoc}
\imagetextslide[0.65]{celsius9.pdf}{\ctof \\ \nl \ftoc}
\imagetextslide[0.65]{celsius10.pdf}{\ctof \\ \nl \ftoc}
\imagetextslide[0.65]{celsius11.pdf}{\ctof \\ \nl \ftoc}
\imageslide[0.65]{celsius12.pdf}
\imageslide[0.65]{celsius13.pdf}
\imageslide[0.65]{celsius14.pdf}
\imageslide[0.65]{celsius15.pdf}
\imageslide[0.65]{celsius16.pdf}


\textslide{
\Large{We can combine networks into larger networks!}
}

\textslide{\Huge{?}}


%%%%% partiality


\textslide{\Large{Cells {\it accumulate information} about a value}}

\imageslide[0.5]{sudoku/sudoku1.png}
\imageslide[0.5]{sudoku/sudoku2.png}
\imageslide[0.5]{sudoku/sudoku3.png}
\imageslide[0.5]{sudoku/sudoku4.png}
\imageslide[0.5]{sudoku/sudoku5.png}
\imageslide[0.5]{sudoku/sudoku6.png}
\imageslide[0.5]{sudoku/sudoku7.png}
\imageslide[0.5]{sudoku/sudoku8.png}
\imageslide[0.5]{sudoku/sudoku9.png}
\imageslide[0.5]{sudoku/sudoku10.png}
\imageslide[0.5]{sudoku/sudoku11.png}
\imageslide[0.5]{sudoku/sudoku12.png}
\imageslide[0.5]{sudoku/sudoku13.png}



\begin{frame}
\begin{center}
\Huge $\{True,\ False\}$
\end{center}
\end{frame}


\begin{frame}
TODO set intersection examples
\end{frame}



\imageslide{badd12.pdf}

\begin{frame}[fragile]

\begin{haskellcode}
data Perhaps a = Unknown | Known a | Contradiction
\end{haskellcode}

\pnl

\begin{haskellcode}
instance Eq a => BoundedJoinSemiLattice (Perhaps a) where

  bottom = Unknown

  (\/) Unknown x           = x
  (\/) x       Unknown     = x
  (\/) Contradiction _     = Contradiction
  (\/) _     Contradiction = Contradiction
  (\/) (Known a) (Known b) =
    if a == b
      then Known a
      else Contradiction
\end{haskellcode}

\end{frame}


\begin{frame}

\begin{columns}
\column{0.5\textwidth}
\includegraphics[scale=0.65]{flat.pdf}
\pause
\column{0.2\textwidth}
\column{0.3\textwidth}
\includegraphics[scale=0.65]{more-information.pdf}
\end{columns}
\end{frame}


\imageslide{perhaps1.pdf}
\imageslide{perhaps2.pdf}


\imageslide[0.6]{doubleplus4.pdf}
\imageslide[0.6]{doubleplus5.pdf}
\imageslide[0.6]{doubleplus6.pdf}
\imageslide[0.6]{doubleplus7.pdf}
\imageslide[0.6]{doubleplus8.pdf}


\begin{frame}
\begin{center}
\Huge $[1,5]$
\end{center}
\end{frame}


\begin{frame}
\begin{center}
\Huge $[1, 5] \cup [2, 7] = [2,5]$

\pnl

\Huge $[2,5] + [9,10] = [11,15]$
\end{center}
\end{frame}


\imageslide{intervaladd1.pdf}
\imageslide{intervaladd2.pdf}

\textslide{(more interval stuff)}

\textslide{
  What other bounded join-semilattices are there?
}

\imageslide[0.7]{powerset.pdf}
\imageslide[0.7]{powerset-upside-down.pdf}

\textslideleft{

  \begin{itemize}
    \item Set intersection or union
    \item Interval intersection
    \item {\tt Perhaps}
  \end{itemize}

  \nl

  And so many more!

  \pnl

  \centering
  {\Huge ?}
}


\textslide{
  What happens when we hit contradiction?
  
  \pnl
  
  {\Huge{:(}}

}

\textslide{
  \Large{
    If we track the provenance of information, \\
    we can help identify the source of contradiction

    \pnl

    Then we can keep track of which subsets of the information are consistent

    \nl

    and which are inconsistent
  }
}

\textslideleft{
  \begin{columns}[T]
    \column{0.05\textwidth}
    \column{0.35\textwidth}
    \Large

    $ [2,5] \cap [3,7] \cap [6,9] = []$

    \pnl

    $[2,5] \cap [3,7] = [3,5]$

    \pnl

    $[3,7] \cap [6,9] = [6,7]$

    \pnl

    $[2,5] \cap [6,9] = []$

  \column{0.3\textwidth}
  \pause
  \centering
    Consistent subsets:

    $\{\}$
    
    $\{[2,5]\}$
    
    $\{[3,7]\}$
    
    $\{[6,9]\}$

    $\{[2,5], [3,7]\}$

    $\{[3,7], [6,9]\}$

    \nl

    Maximal consistent subsets:

    $\{[2,5], [3,7]\}$

    $\{[3,7], [6,9]\}$

    \column{0.3\textwidth}
    \pause
    \centering
    Inconsistent subsets:

    $\{[2,5], [6,9]\}$
    $\{[2,5], [3,7], [6,9]\}$

    \nl

    Minimal inconsistent subsets:

    $\{[2,5], [6,9]\}$
  \end{columns}
}

\textslide{
  \Large{
    This concept is something called a {\it Truth Management System}
  }
}

\textslide{
  Now that we can handle contradiction, we can make guesses!

  \pnl

  This lets us encode search problems easily

  \nl

  \def\svgwidth{0.6\columnwidth}
  \input{coloured-map.pdf_tex}
}



\textslide{\centering {\Huge ?}}


\textslide{
  We can relax some of our conditions in certain circumstances
}

\imageslide[0.8]{dcpo.pdf}

\textslide{
  We can turn any expression tree into a propagator network

  There will only ever be one writer to a cell

  \nl

  \LARGE{$(5+2) \times (x+y)$}

  \includegraphics[scale=0.5]{tree.pdf}
}

\textslide{\Huge{Wrapping up}}


\begin{frame}

Alexey Radul's work on propagators:

\begin{itemize}
\item Art of the Propagator \\
      \url{http://web.mit.edu/~axch/www/art.pdf}
\item Propagation Networks: A Flexible and Expressive Substrate for Computation \\
      \url{http://web.mit.edu/~axch/www/phd-thesis.pdf}
\end{itemize}
\end{frame}


\textslideleft{

Lindsey Kuper's work on LVars is closely related, and works today:

\begin{itemize}
\item Lattice-Based Data Structures for Deterministic Parallel and Distributed Programming \\
      \url{https://www.cs.indiana.edu/~lkuper/papers/lindsey-kuper-dissertation.pdf}
\item lvish library \\
      \url{https://hackage.haskell.org/package/lvish}
\end{itemize}

}

\textslideleft{
Edward Kmett has worked on:

\begin{itemize}
\item Making propagators go fast
\item Scheduling strategies and garbage collection
\item Relaxing requirements (Eg. not requiring a full join-semilattice, admitting non-monotone functions)
\end{itemize}

Ed's stuff:
\begin{itemize}
\item \url{http://github.com/ekmett/propagators}
\item \url{http://github.com/ekmett/concurrent}
\item Lambda Jam talk (Normal mode): \\
      \url{https://www.youtube.com/watch?v=acZkF6Q2XKs}
\item Boston Haskell talk (Hard mode): \\
      \url{https://www.youtube.com/watch?v=DyPzPeOPgUE}

\end{itemize}
}

\textslideleft{

In conclusion, propagator networks:

\begin{itemize}
\item Admit any Haskell function you can write today \ldots
\item \ldots and more functions!
\item compute bidirectionally
\item give us constraint solving and search
\item mix all this stuff together
\item parallelise and distribute
\end{itemize}
}


\textslide{\Large{Thanks for listening!}}


\end{document}

